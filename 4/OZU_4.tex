% !TeX encoding = UTF-8

\documentclass{protokol}

\usepackage{tikz}
\usetikzlibrary{calc}
\usetikzlibrary{arrows}

%====== Units =====
\usepackage{siunitx}
\sisetup{inter-unit-product =\ensuremath{\cdot}}
\sisetup{group-digits = integer}
\sisetup{output-decimal-marker = {,}}
\sisetup{exponent-product = \ensuremath{\cdot}}
\sisetup{separate-uncertainty}
\sisetup{tight-spacing = false}
%\sisetup{scientific-notation = true}
%\sisetup{round-mode=places,round-precision=4}
%\sisetup{evaluate-expression}


%====== Grafy =====
\usepackage{pgfplots}
\pgfplotsset{width=0.8\linewidth, compat=1.17}
\def\plotcscale{0.8}
\usepackage{pgfplotstable}
\usepackage[figurename=Graf]{caption} % figure caption rename
%====== Rovnice align block ======
\usepackage{amsmath}
\setlength{\jot}{10pt} % rozestup mezi řádky

\graphicspath{ {./img/} }

%====== Vyplňte údaje ======
\jmeno{Jakub Charvot}
\kod{240844}
\rocnik{2.}
\obor{MET}
\skupina{MET/4}
\spolupracoval{Radek Kučera}

\merenodne{10.\,11.\,2022}
\odevzdanodne{24.\,11.\,2022}
\nazev{Operační usměrňovače}
\cislo{4} %měřené úlohy

\predmet{Analogové elektronické obvody}
\ustav{Ústav mikroelektroniky}
\skola{FEKT VUT v Brně}

\def\para{x+0}
\def\parb{\para-80}

% CSV
\usepackage{blindtext}

\usepackage{subfiles} % Best loaded last in the preamble
\usepackage{datatool}

% \DTLloaddb[omitlines=1]{data}{data/data-voda.csv}
% \DTLloaddb[omitlines=2]{druha}{data/druha-cast.csv}



\begin{document}
%====== Vygenerování tabulky ======
%	\maketitle

\section{Zpracování měřených hodnot}	
\subsection{Sériový odpor (\(R_{ESR} \) )}
\begin{table}[h!]
	\centering
	\def\arraystretch{1.4}
	\begin{tabular}{|c|c|c|c|c|}
		\hline
		Kondenzátor & \(\Delta U\)\ [\unit{\milli\volt}]  & \(I_{vyb}\ [\unit{\ampere}] \)  & \(R_{ESR}\ [\unit{\milli\ohm}] \)  & Tabulkové \(R_{ESR}\ [\unit{\milli\ohm}] \)  \\ \hline \hline
		Nichicon  & 87,5                                         & 2                           & 43,75                      & 200                                   \\ \hline
		Eaton      & 30,6                                         & 2                           & 15,30                      & 34                                    \\ \hline
		Maxwell       & 43,7                                         & 2                           & 21,85                      & 75                                    \\ \hline
	\end{tabular}
	\caption{Měřené a vypočtené hodnoty - \(R_{ESR} \) }
	\label{tab:ojjoij}
\end{table}

\[
	R_{ESR} = \frac{\Delta U}{I_{vyb} } = \frac{87,5}{2}=\qty{43,75}{\milli\ohm}
\]

\subsection{Měření kapacity}
Vyjdeme z následujícího vztahu:
\[
	C=\frac{I_{vyb} (t_{2} -t_{1} )}{U_{2} -U_{1} }
\]
Při zvoleném poklesu napětí o 1 V a vybíjecím proudu 1 A odpovídá kapacita \(C\)  přímo změřenému času poklesu. 

\begin{table}[h!]
	\centering
	\def\arraystretch{1.4}
	\begin{tabular}{|c|c|}
		\hline
		Kondenzátor & \(C\ [\unit{\farad}] \)  \\ \hline \hline
		Nichicon  & 8,86 \\ \hline
		Eaton     & 8,34 \\ \hline
		Maxwell   & 9,2 \\ \hline
	\end{tabular}
	\caption{Skutečné měřené kapacity kondezátorů 10 F. }
	\label{tab:ojjoij}
\end{table}

\subsection{Měření svodového proudu \(I_{L} \) (samovybíjení)}
\begin{table}[h!]
	\centering
	\def\arraystretch{1.4}
	\begin{tabular}{|c|c|c|c|c|c|}
		\hline
		Kondenzátor & \(U_{1} \)\ [\unit{\volt}]  & \(U_{2}\ [\unit{\volt}] \)  & \(t\ [\unit{\second}] \)  & \(I_{L}\ [\unit{\micro\ampere}] \) & Tab. \(I_{L}\ [\unit{\micro\ampere}] \)  \\ \hline \hline
		Nichicon  & 2,215 & 2,160  & 899 & 542,048 & 5000 \\ \hline
		Eaton     & 2,212 & 2,114 & 900  & 908,133 & 23 \\ \hline
		Maxwell   & 2,200 & 2,109 & 897  & 933,333 & 30 \\ \hline
	\end{tabular}
	\caption{Porovnání vypočtených a tabulkových hodnot \(I_{L} \) .}
	\label{tab:ojjoij}
\end{table}

\[
	I_{L} = \frac{C(U_{1} - U_{2} )}{t}=\frac{8,86(2,215 - 2,16 )}{899}=\qty{542,048}{\micro\ampere}
\]

\clearpage
\section{Závěr}
	Pro ideální vlastnosti superkondenzátoru chceme co nejmenší sériový odpor \(R_{ESR} \), protože právě jeho existence způsobuje ztráty při každém nabíjení a vybíjení. Stejně tak chceme co nejmenší samovybíjecí proud, naopak hledáme co nejvyšší kapacitu. 
	
	Každý z dodaných kondezátorů vede v jednom z těchto parametrů. V tabulkových i reálných hodnotách \(R_{ESR} \) vede značka Eaton, přičemž u všech vzorků byla měřená hodnota výrazně nižší než tabulková. 

	Přestože se ve všech případech jedná o superkondenzátory s deklarovanou kapacitou 10 F, naměřili jsme vždy méně. Nejvíce tedy 9,2 F u značky Maxwell. Otázkou ovšem je, jaké byly tyto hodnoty v čase, kdy byly součástky nové a nepodléhaly dlouhodobému vystavení studentům. 

	Nejnižší samovybíjecí proud \(I_{L} \) jsme naměřili u značky Nichicon a to zhruba 0,5 mA, přičemž specifikace uváděla desetkrát více. Zbylé kondenzátory měli vybíjecí proudy téměř dvojnásobné, i když jejich specifikace udávaly hodnoty více než o řád menší.
\end{document}