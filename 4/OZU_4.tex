% !TeX encoding = UTF-8

\documentclass{protokol}

\usepackage{tikz}
\usetikzlibrary{calc}
\usetikzlibrary{arrows}

%====== Units =====
\usepackage{siunitx}
\sisetup{inter-unit-product =\ensuremath{\cdot}}
\sisetup{group-digits = integer}
\sisetup{output-decimal-marker = {,}}
\sisetup{exponent-product = \ensuremath{\cdot}}
\sisetup{separate-uncertainty}
\sisetup{tight-spacing = false}
%\sisetup{scientific-notation = true}
%\sisetup{round-mode=places,round-precision=4}
%\sisetup{evaluate-expression}


%====== Grafy =====
\usepackage{pgfplots}
\pgfplotsset{width=0.8\linewidth, compat=1.17}
\def\plotcscale{0.8}
\usepackage{pgfplotstable}
\usepackage[figurename=Graf]{caption} % figure caption rename
%====== Rovnice align block ======
\usepackage{amsmath}
\setlength{\jot}{10pt} % rozestup mezi řádky

\graphicspath{ {./img/} }

%====== Vyplňte údaje ======
\jmeno{Jakub Charvot}
\kod{240844}
\rocnik{2.}
\obor{MET}
\skupina{MET/4}
\spolupracoval{Radek Kučera}

\merenodne{10.\,11.\,2022}
\odevzdanodne{24.\,11.\,2022}
\nazev{Operační usměrňovače}
\cislo{4} %měřené úlohy

\predmet{Analogové elektronické obvody}
\ustav{Ústav mikroelektroniky}
\skola{FEKT VUT v Brně}

\def\para{x+0}
\def\parb{\para-80}

% CSV
\usepackage{blindtext}

\usepackage{subfiles} % Best loaded last in the preamble
\usepackage{datatool}

% \DTLloaddb[omitlines=1]{data}{data/data-voda.csv}
% \DTLloaddb[omitlines=2]{druha}{data/druha-cast.csv}



\begin{document}
%====== Vygenerování tabulky ======
%	\maketitle

\section{Zpracování měřených hodnot}

\clearpage
\section{Závěr}
	V prvním kroku jsme nepřímo měřili tepelnou kapacitu kalorimetru, na základě výše uvedeného výpočtu jsme došli k hodnotě \(C_k \doteq \qty{51,244}{\joule\per\kelvin}\), tuto hodnotu jsme následně použili i v dalších výpočtech, pro dosažení co nejpřesnějších výsledků. 
\\

	Dále jsme analyzovali tepelné ztráty kalorimetru. Již během ohřevu docházelo k významným ztrátám energie. Zde se projevuje více faktorů, jednak samotné tepelné ztráty kalorimetru, druhak se část výkonu zdroje nepochybně ztrácí na přívodních kabelech a neohřívá tedy přímo vodu. Výpočty neusnadňuje ani fakt, že tepelná kapacita kalorimetru a objem vody v něm se v průběhu měření mění z důvodu přesunu vody z vnitřní nádoby do prostoru mezi stěny a izolaci. Pro jeho složitost jsme ale tento proces ve výpočtech zanedbali.
\\

	Samotné tepelné ztráty jsou viditelné po vypnutí ohřevu, zde jsme dospěli k hodnotě ztrátového výkonu \qty{5,693}{\joule\per\second}. Lze předpokládat, že s klesajícím rozdílem teplot by toto číslo klesalo, ale i tak se jedná o poměrně vysoké ztráty. Jejich omezení by pomohla zejména lepší izolace v prostoru víka nádoby.
\\

	Při porovnání rychlosti ohřevu samotné vody a vody s parafinovými polštářky jsme nezaznamenali velké rozdíly, obě závislosti mají přibližně lineární charakter. U parafinu je možné vidět mírný pokles rychlosti ohřevu při dosažení teploty zhruba \qty{45}{\degreeCelsius}, kdy se část energie začně spotřebovávat na tání parafinu, ten je amorfní látkou, proto taje postupně v širším rozsahu teplot. Proč je ale možné vidět stejný "schod" i u samotné vody nám bohužel není jasné. 
\end{document}