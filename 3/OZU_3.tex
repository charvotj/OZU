% !TeX encoding = UTF-8

\documentclass{protokol}

\usepackage{tikz}
\usetikzlibrary{calc}
\usetikzlibrary{arrows}

%====== Units =====
\usepackage{siunitx}
\sisetup{inter-unit-product =\ensuremath{\cdot}}
\sisetup{group-digits = integer}
\sisetup{output-decimal-marker = {,}}
\sisetup{exponent-product = \ensuremath{\cdot}}
\sisetup{separate-uncertainty}
\sisetup{tight-spacing = false}
%\sisetup{scientific-notation = true}
%\sisetup{round-mode=places,round-precision=4}
%\sisetup{evaluate-expression}


%====== Grafy =====
\usepackage{pgfplots}
\pgfplotsset{width=0.8\linewidth, compat=1.17}
\def\plotcscale{0.8}
\usepackage{pgfplotstable}
\usepackage[figurename=Graf]{caption} % figure caption rename
%====== Rovnice align block ======
\usepackage{amsmath}
\setlength{\jot}{10pt} % rozestup mezi řádky

\graphicspath{ {./img/} }

%====== Vyplňte údaje ======
\jmeno{Jakub Charvot}
\kod{240844}
\rocnik{2.}
\obor{MET}
\skupina{MET/4}
\spolupracoval{Radek Kučera}

\merenodne{10.\,11.\,2022}
\odevzdanodne{24.\,11.\,2022}
\nazev{Operační usměrňovače}
\cislo{4} %měřené úlohy

\predmet{Analogové elektronické obvody}
\ustav{Ústav mikroelektroniky}
\skola{FEKT VUT v Brně}

\def\para{x+0}
\def\parb{\para-80}

% CSV
\usepackage{blindtext}

\usepackage{subfiles} % Best loaded last in the preamble
\usepackage{datatool}

\DTLloaddb[omitlines=1]{data}{data/data-voda.csv}
% \DTLloaddb[omitlines=2]{druha}{data/druha-cast.csv}



\begin{document}
%====== Vygenerování tabulky ======
%	\maketitle

\section{Zpracování měřených hodnot}
\subsection{Stanovení tepelné kapacity kalorimetru}

\[
	m_1 \cdot c_v \cdot\left(T_1-T\right)=m_2 \cdot c_v \cdot\left(T-T_2\right)+C_k \cdot\left(T-T_2\right)
\]
\[
	C_k=\frac{m_1 \cdot c_v \cdot\left(T_1-T\right)}{\left(T-T_2\right)}-m_2 \cdot c_v
\]
\[
	C_k=\frac{0,088 \cdot 4180 \cdot\left(78-50\right)}{\left(50-23\right)}-0,079 \cdot 4180
\]
\[
	C_{k} \doteq \qty{51,244}{\joule\per\kelvin}
\]

\subsection{Tepelné ztráty kalorimetru a ohřev vody}
\begin{figure*}[h!]
	\begin{tikzpicture}
		\centering
		\begin{axis}
			[
			xlabel={\( t\ [\unit{\second}]\)},
			ylabel={\( \theta\ [\unit{\degreeCelsius}]\)},
			%axis y line*=left, % dve y osy
			width=1\textwidth,
			height = 0.5\textwidth,
			legend pos=south east,
%			xmin=0,
%			ymin=0,
%			xmax=100
%			ymax=100
			]

			\addplot[mark=x, mark options={solid}, thin, blue, only marks, mark size=3pt] table [skip first n=1, x=cas, y=teplota, col sep=comma, row sep=newline] {data/data-voda.csv};
			\addlegendentry{Voda}
			
			\addplot[mark=+, mark options={solid}, thin, red, only marks, mark size=3pt] table [skip first n=1, x=cas, y=teplota, col sep=comma, row sep=newline] {data/data-parafin.csv};
			\addlegendentry{Voda s parafinovými polštářky}
		\end{axis}   
	\end{tikzpicture}
	\caption{Závislost teploty vody v kalorimetru na čase ohřevu popř. ustálení.}
\end{figure*}

\subsubsection{Ohřev vody}
Dodaná energie: 
 \[
	W_{in}=U\cdot I \cdot t_{on}  = 9,8\cdot 3\cdot 990 = \qty{29106}{\joule}
 \]
Teor. energie spotřebovaná na ohřev vody: 
 \[
	Q=(m_{v} \cdot c_{v}+C_{k} ) \cdot (T_{max}-T_{init}) = (0,143 \cdot 4180 + 51,244) \cdot (62-23) = \qty{25310.376}{\joule}
 \]
 Teor. čas ohřevu vody: 
 \\
 (Skutečný čas \qty{990}{s})
 \[
	t_{t} = \frac{Q}{U\cdot I}=\frac{25310,376}{9,8\cdot 3}\doteq\qty{861}{\second}
 \]
Ztráty při ohřevu: 
\[
	W_{zt} = W_{in} -Q= \num{29106}-\num{25310.376}=\qty{3795,624}{\joule}
\]
Ztrátový výkon: 
\[
	P_{zt} =  \frac{W_{zt}}{t}=\frac{25310.376}{1050}=\qty{24,105}{\joule\per\second}
\]

\subsubsection{Samovolné chladnutí vody }
Ztracené teplo: 
\[
	Q_{zt} = (m_{v} \cdot c_{v}+C_{k} ) \cdot (T_{end}-T_{max}) = (0,143 \cdot 4180 + 51,244) \cdot (57-62) = \qty{-3244.92}{\joule}
\]
Ztrátový výkon: 
\[
	P_{zt} =  \frac{Q_{zt}}{t}=\frac{3244.92}{570}=\qty{5,693}{\joule\per\second}
\]

\subsubsection{Ohřev vody s parafinovými polštářky}

\begin{figure*}[h!]
	\begin{tikzpicture}
		\centering
		\begin{axis}
			[
			xlabel={\( t\ [\unit{\second}]\)},
			ylabel={\( \theta\ [\unit{\degreeCelsius}]\)},
			%axis y line*=left, % dve y osy
			width=1\textwidth,
			height = 0.5\textwidth,
			legend pos=north west,
%			xmin=0,
%			ymin=0,
%			xmax=100
%			ymax=100
			]

			\addplot[mark=x, mark options={solid}, thick, blue, only marks, mark size=3pt] table [skip first n=1, x=cas, y=teplota, col sep=comma, row sep=newline] {data/data-parafin.csv};
			
		\end{axis}   
	\end{tikzpicture}
	\caption{Závislost teploty vody a parafinových polštářků v kalorimetru na čase ohřevu.}
\end{figure*}

\subsubsection{Srovnání rychlostí ohřevu}
Samotná voda:
\[
	\frac{\Delta T}{\Delta t}\cdot60= \frac{60-23}{990}\cdot60=\qty{2,24}{\degreeCelsius\per\minute}
\]
Voda s parafinovými polštářky:
\[
	\frac{\Delta T}{\Delta t}\cdot60= \frac{60-29}{750}\cdot60=\qty{2,48}{\degreeCelsius\per\minute}
\]

	\clearpage
\section{Závěr}
	V prvním kroku jsme nepřímo měřili tepelnou kapacitu kalorimetru, na základě výše uvedeného výpočtu jsme došli k hodnotě \(C_k \doteq \qty{51,244}{\joule\per\kelvin}\), tuto hodnotu jsme následně použili i v dalších výpočtech, pro dosažení co nejpřesnějších výsledků. 
\\

	Dále jsme analyzovali tepelné ztráty kalorimetru. Již během ohřevu docházelo k významným ztrátám energie. Zde se projevuje více faktorů, jednak samotné tepelné ztráty kalorimetru, druhak se část výkonu zdroje nepochybně ztrácí na přívodních kabelech a neohřívá tedy přímo vodu. Výpočty neusnadňuje ani fakt, že tepelná kapacita kalorimetru a objem vody v něm se v průběhu měření mění z důvodu přesunu vody z vnitřní nádoby do prostoru mezi stěny a izolaci. Pro jeho složitost jsme ale tento proces ve výpočtech zanedbali.
\\

	Samotné tepelné ztráty jsou viditelné po vypnutí ohřevu, zde jsme dospěli k hodnotě ztrátového výkonu \qty{5,693}{\joule\per\second}. Lze předpokládat, že s klesajícím rozdílem teplot by toto číslo klesalo, ale i tak se jedná o poměrně vysoké ztráty. Jejich omezení by pomohla zejména lepší izolace v prostoru víka nádoby.
\\

	Při porovnání rychlosti ohřevu samotné vody a vody s parafinovými polštářky jsme nezaznamenali velké rozdíly, obě závislosti mají přibližně lineární charakter. U parafinu je možné vidět mírný pokles rychlosti ohřevu při dosažení teploty zhruba \qty{45}{\degreeCelsius}, kdy se část energie začně spotřebovávat na tání parafinu, ten je amorfní látkou, proto taje postupně v širším rozsahu teplot. Proč je ale možné vidět stejný "schod" i u samotné vody nám bohužel není jasné. 
\end{document}